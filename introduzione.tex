Il metaverso \`e un ambiente virtuale in cui l'utente ha la possibilità di creare e gestire identità digitali, creare asset e scambiarli con altri utenti. Esso sta dando vita ad un modo nuovo con cui le persone possono interagire e socializzare.
%
In questa realtà, \nomeapp{} vuole essere per l'utente che ne usufruisce una possibilità di interagire con altri utenti all'interno del metaverso di Ringmaster.
%
\nomeapp{}, infatti, \`e un gioco di gare equestri in 3D sviluppato con Unreal Engine 5.
%
In questo gioco ogni utente con il proprio cavallo ha la possibilità di gareggiare contro i cavalli degli altri utenti, ognugno dei quali con i propri metadati. 
% Da aggiungere il numero cavalli e altre info
%
Ogni cavallo \`e associato ad un NFT \textit{(Non-Fungible Token)}, il quale costituisce il suo \textit{certificato di proprietà}. Questo permette ai giocatori di scambiare i cavalli posseduti o anche di utilizzarli al di fuori di \nomeapp{}.
%
La proprietà digitale \`e garantita dall'NFT in quanto questa nuova forma di criptovaluta si basa sulla tecnologia blockchain. Infatti, questa tecnologia permette la creazione di un database condiviso con la caratteristica principale di essere immutabile, cio\'e garantisce che i dati registrati su di essa non possano essere modificati in alcun modo. 

Il gioco sfrutta un'architettura client-server e una metodologia di sviluppo MVC \textit{(Model-View-Controller)}. 
%
Lo svolgersi della gara \`e gestito da un server che decide, grazie ad un algoritmo, i tempi di ciascun cavallo partecipante sia sulla base dei propri metadati (come la velocità massima, l'accellerazione e la resistenza) sia su fattori casuali in modo che il risultato non sia prestabilito.
% Predeterminato o Prestabilito 
%
Il client e il server comunicano tra loro con lo standard Json. Le comunicazioni passano attraverso una WebSocket e viene usato un paradigma di programmazione ad eventi.
%Non so come aggiungere meglio che uso la program. ad eventi

Infine, molti degli asset presenti nel gioco sono stati creati con il software di modellazione 3D Blender.