Questa tesi tratta l'analisi, la progettazione e l'implementazione di Metarace, un'applicazione in Unreal Engine 5 sviluppata per il Metaverso.
%
Poiché il Metaverso necessita ancora di nuove tecnologie e nuovistrumenti per essere realizzato, verrà discusso cosa si intende quando si parla di Metaverso oggi e cosa significa sviluppare in quest'ottica.
%
Metarace si avvicina a questo concetto costruendo un mondo interattivo al quale gli utenti possono accedere attraverso un avatar digitale.
%
Metarace, infatti, è un gioco in cui gli utenti hanno la possibilità di possedere dei cavalli, di farli partecipare a gare di velocità e di assistervi attraverso un visore per la realtà virtuale.
%
L'esito della gara verrà stabilito da un server sulla base di caratteristiche intrinseche del cavallo, come la velocità, la resistenza e la potenza insieme ad una componente casuale che garanstisce la non prevedibilità del risultato.
%
Questo lavoro tratta la realizzazione dell'intera architettura di Metarace, tra cui la logica implementativa di base, lo strato del network, l'interfaccia utente, la creazione del mondo virtuale e degli oggetti 3D.
%
Il mondo di gioco è stato realizzato con il motore grafico Unreal Engine 5 utilizzato in combinazione con la suite 3D Blender.
%
Il software è stato progettato seguendo l'approccio della programmazione ad eventi e sono state messe in atto tecniche per sostenere un basso livello di accoppiamento e un alto livello di coesione nel codice.
%
La comunicazione tra gli utenti e il server è stata implementata attraverso un'architettura client-server, utilizzando il protocollo WebSocket per poter tenere attive delle connessioni di tipo full-duplex bidirezionali con più giocatori.
%
Durante la trattazione verrà posta particolare attenzione a tutti gli strumenti utilizzati durante l'implementazione ed alla traduzione del modello di dominio in codice.
%
Metarace si andrà ad inserire in un contesto più ampio di applicazioni con cui comporrà l'universo virtuale dell'azienda Ringmaster.


%Il metaverso \`e un ambiente virtuale in cui l'utente ha la possibilità di creare e gestire identità digitali, creare asset e scambiarli con altri utenti. Esso sta dando vita ad un modo nuovo con cui le persone possono interagire e socializzare.
%
%In questa realtà, \nomeapp{} vuole essere per l'utente che ne usufruisce una possibilità di interagire con altri utenti all'interno del metaverso di Ringmaster.
%
%\nomeapp{}, infatti, \`e un gioco di gare equestri in 3D sviluppato con Unreal Engine 5.
%
%In questo gioco ogni utente con il proprio cavallo ha la possibilità di gareggiare contro i cavalli degli altri utenti, ognugno dei quali con i propri metadati. 
% Da aggiungere il numero cavalli e altre info
%
%Ogni cavallo \`e associato ad un NFT \textit{(Non-Fungible Token)}, il quale costituisce il suo \textit{certificato di proprietà}. Questo permette ai giocatori di scambiare i cavalli posseduti o anche di utilizzarli al di fuori di \nomeapp{}.
%
%La proprietà digitale \`e garantita dall'NFT in quanto questa nuova forma di criptovaluta si basa sulla tecnologia blockchain. Infatti, questa tecnologia permette la creazione di un database condiviso con la caratteristica principale di essere immutabile, cio\'e garantisce che i dati registrati su di essa non possano essere modificati in alcun modo. 

%Il gioco sfrutta un'architettura client-server e una metodologia di sviluppo MVC \textit{(Model-View-Controller)}. 
%
%Lo svolgersi della gara \`e gestito da un server che decide, grazie ad un algoritmo, i tempi di ciascun cavallo partecipante sia sulla base dei propri metadati (come la velocità massima, l'accellerazione e la resistenza) sia su fattori casuali in modo che il risultato non sia prestabilito.
% Predeterminato o Prestabilito 
%
%Il client e il server comunicano tra loro con lo standard Json. Le comunicazioni passano attraverso una WebSocket e viene usato un paradigma di programmazione ad eventi.
% Non so come aggiungere meglio che uso la program. ad eventi

%Infine, molti degli asset presenti nel gioco sono stati creati con il software di modellazione 3D Blender.

% \section*{Obiettivi}

% Descrizione ad alto livello di tutta la tesi, un prologo