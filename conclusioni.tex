Metarace riesce a creare un ambiente virtuale immersivo grazie all'utilizzo di Unreal Engine e di un visore per la realtà virtuale.
%
Il giocatore può muoversi all'interno di questo mondo per assistere alla gara da diversi punti di osservazione, che coincidono con gli spalti all'interno del gioco.
%
All'interno del mondo di gioco sono state create delle scenografie digitali che hanno come obiettivo quello di rendere il gioco simile ad un'esperienza reale.
%
Ho voluto però mantenere uno stile grafico low poly per dargli una caratterizzazione particolare.
%
In questa tesi mi sono poi soffermato sulle tecnologie e gli strumenti utilizzati per la sua realizzazione, a partire dalle tecniche di modellazione 3D, passando per la progammazione di codice coeso e poco accoppiato, fino ad arrivare all'implementazione del paradigma di programmazione ad eventi. 
%
Lo strumento principale per la realizzazione di Metarace, Unreal Engine 5, si è dimostrato un software affidabile e completo per la realizzazione degli obiettivi posti inizialmente.
%
L'utilizzo del linguaggio di programmazione C++ unito all'elevata modularità del codice sorgente di Unreal Engine offre la possibilità di scegliere quali parti di architettura dell'engine utilizzare e quali invece scartare perché superflue.
%
Grazie inoltre al software di modellazione Blender è stato possibile creare e modificare oggetti e animazioni digitali che hanno contribuito alla costruzione del mondo 3D.

Il mondo 3D di Metarace è uno dei mondi all'interno dell'universo virtuale dell'azienda Ringmaster. 
%
Questa caratteristica, unita al fatto che può essere sperimentato attraverso un avatar digitale e un visore per la realtà 3D, e che i contenuti del mondo di gioco dovranno essere altamente personalizzabili, gli conferisce i presupposti per essere incluso in una futura implementazione del Metaverso per come è stato definito.
%
Il Metaverso, per come è stato immaginato e concettualizzato nella letteratura Cyberpunk prima e da Matthew Ball successivamente, comporterebbe l'arrivo di una nuova era tecnologica che andrebbe a rivoluzionare il modo con cui l'uomo si approccia alla tecnologia.
%
La realizzazione del Metaverso necessiterebbe di una nuova iterazione di Internet, di nuovi protocolli ma soprattutto di nuove tecnologie che ancora non sono state concepite.
%
Per questo motivo non possiamo sapere, ad oggi, se il Metaverso verrà mai realizzato nella sua concezione originaria.
%
Nel frattempo quello che si può fare è progettare applicativi altamente scalabili in modo che siano adattabili alla prossima innovazione tecnologica.