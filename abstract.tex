Il metaverso è un ambiente virtuale in cui l'utente ha la possibilità di creare e gestire identità digitali, creare asset e scambiarli con altri utenti. Esso sta dando vita ad un modo nuovo con cui le persone possono interagire e socializzare. \\
In questa realtà Metarace vuole essere una possibilità che l'utente che si affaccia nel metaverso di Ringmaster ha per interagire con altri utenti. Esso è un gioco 3D sviluppato con Unreal Engine 5 di corse di cavalli. \\
I giocatori potranno possedere dei cavalli e farli gareggiare con quelli di altri giocatori. I cavalli saranno associati ad un NFT che costituirà il loro "certificato di proprietà" che permetterà ai giocatori di scambiarli o di utilizzarli al di fuori del metarace. 
La proprietà digitale è garantita dall'NFT in quanto questa nuova forma di criptovaluta si basa sulla tecnologia blockchain. Questa tecnologia permette la creazione di un database condiviso tra diversi computer collegati in rete con la caratteristica principale essere di essere immutabile, cioè garantisce che i dati registrati su di essa non possano essere modificati in alcun modo. \\
Il gioco sfrutta un'architettura client-server e una metodologia di sviluppo MVC. Lo svolgersi della gara sarà gestito da un server che grazie ad un algoritmo deciderà i tempi di ciascun cavallo partecipante sia sulla base dei propri metadati (come la velocità massima, l'accellerazione e la resistenza) sia su fattori casuali in modo che il risultato non sia scontato. Il client e il server comunicheranno tra di loro con lo standard Json. Le comunicazioni passano attraverso una WebSocket e viene usato un paradigma di programmazione ad eventi. \\
Molti degli asset presenti nel gioco sono stati creati con il software di modellazione 3D Blender. 