% Cosa vuoi dire Emanuele? è l'abstract di un progetto perciò non puoi andare troppo sul filosofico. Scrivi che il metaverso è una figata ma che ha bisogno degli NFTs per non essere un semplice gioco. Noi abbiamo fatto un gioco e lo abbiamo pensato con l'utilizzo di nft, perché? Perché gli nft sono un contratto che si porta dietro tutta una serie di "garanzie" che non si pensava si potessero avere all'interno del mondo virtuale. Perciò vanno così a braccetto con il metaverso, lui è letteralmente un mondo virtuale in cui le persone possiedono delle cose e interagiscono tra di loro, avere un certificato di proprietà indipendente però dalla persone e non modificabile è un'assicurazione incredibile.
% La tesi verte sulle tecnologie utilizzate, quindi Unreal Engine 5 ma anche Blender, le websocket e la programmazione ad eventi. 

In questa tesi verrà trattato il progetto \nomeapp{}, un gioco di competizioni equestri.
%
Lo scopo di questo progetto è quello di creare un mondo virtuale immersivo, fruibile con il visore di realtà virtuale, in cui gli utenti possono diventare proprietari di cavalli digitali, iscriverli a competizioni equestri ed assistervi attraverso i propri avatar.
%
Il gioco è sviluppato con il motore grafico Unreal Engine 5 in combinazione con il software di modellazione Blender, verranno quindi descritte alcune funzionalità di questi due software e di come è possibile usarli in combinazione.


% Versione in italiano:

% In questa tesi verrà trattato il progetto Metarace, un crypto-game di competizioni equestri fruibile nel metaverso. Lo scopo di questo progetto è quello di creare una realtà virtuale immersiva in cui gli utenti possono diventare proprietari di cavalli digitali, iscriverli a competizioni equestri ed assistervi attraverso i propri avatar. La proprietà dei cavalli sarà garantita da un NFT, un token non fungibile basato sulla tecnologia blockchain che andrà a costituire il suo certificato di autenticità. Il gioco, infatti, si baserà sul paradigma Play-to-earn: gli oggetti digitali di gioco che l'utente otterrà in caso di vittoria saranno accompagnati dai propri NFTs. Metarace andrà ad inserirsi in un contesto più ampio di applicazioni con le quali comporrà il metaverso.

% In questa tesi verrà trattato il progetto Metarace, un gioco di competizioni equestri fruibile nel metaverso. Lo scopo di questo progetto è quello di creare una realtà virtuale immersiva in cui gli utenti possono diventare proprietari di cavalli digitali, iscriverli a competizioni equestri ed assistervi attraverso i propri avatar. Verranno descritti alcune funzionalità del motore grafico Unreal Engine e di come è stato usato in questo progetto insieme al software di modellazione Blender. Il progetto si inserisce in un contesto più ampio di applicazioni con le quali comporrà il metaverso. 

% Versione in inglese:

% In this thesis will be treated the Metarace project, a game of equestrial competition usable in the metaverse. The purpuse of this project is to create an immersive virtual reality in wich users can become owners of digital horses, enroll them in equestrial competition and assist them through their avatars. The ownership of the horses will be guaranteed by an NFT, a non-fungible token based on blockchain technology that will constitute its certificate of authenticity. The game, in fact, will be based on the Play-to-earn paradigm: the digital game objects that the user will obtain in case of victory will be accompanied by their own NFTs. Metarace will fit into a wider context of applications that will compose the metaverse.


%L'abstract io lo farei più generale, le tecnologie non le userei neanche nell'abstract. Direi più "abbiamo fatto questo", "abbiamo fatto quest'altro". Parliamo di più sul metaverso. Qui si parlerà di tutte queste cose.
%
% VR = Virtual Reality = ti metti l'headset e vieni immerso in questo mondo virtuale completamente immersivo
% AR = Aumented reality = elementi virtuali vengono messi in sovraimpressione davanti al mondo reale
% Modello di business--> Play-to-earn
%
% Fungibility = The ability of an asset to be exchanged or substituted with similar assets of the same value ==> Un esempio facile da capire sono le monete. Che possiedi 5 monete da 1€ o una banconota da 5€ possiedi 5€ in entrambi i casi. 5 monete da 1€ possono essere scambiate per una banconota da 5€ e non cambia il valore di ciò che possiedi
% Non fungibility perciò è l'esatto opposto. Ogni asset è univoco e non può essere sostituito facilmente con qualcosa di simile. 
% Gli NFT sono delle crypto valute ma a differenza di quelle fungibili (come i bitcoin) sono completamente univoci. Esistono come una stringa di numeri e lettere salvate in una blockchain. Questa informazione può contenere l'origine dell'asset, chi l'ha venduto e quando. E viene anche criptata in modo tale da preservarne l'autenticità e la rarità.
%
% Con il termine blockchain si fa riferimento ad un particolare tipo di distributed ledger, in cui i dati delle transazioni sono raccolti in blocchi collegati secondo una sequenza temporale e questi sono inseriti nel registro. In questo modo le informazioni archiviate, una volta immesse nel registro e validate, sono inalterabili perché ogni nuova informazione è indissolubilmente legata allo storico delle transazioni precedenti: eventuali manipolazioni di dati, pertanto, sarebbero immeditamante evidenziate, non consentendo la successiva validazione. 
%
% Con il termine Distributed Ledger Technologies (DLT) si fa riferimento a "libri mastri" (o registri) elettronici, distribuiti geograficamente su un'ampia rete di nodi, i cui dati sono protetti da potenziali attacchi informatici grazie al fatto che le stesse informazioni sono ridondate, verificate e validate mediante l'adozione di diversi protocolli (o regole) comunemente accettati da ciascun partecipante.
%
% Settore con obsolescenza molto alta. È una rivoluzione sociale.
%
% Cryptokitty è stato il primo tipo di collezionabile che ha portato alla necessità di creare lo standard per la definizione di NFTs
%
% Play to earn: la gente compra gli asset per sfidarsi e chi vince vince altri asset monetalizzabili. La gente che gioca seriamente guadagna tanti soldi. Per giocare ti servono gli asset che hanno dei costi iniziali. Ora ci sono le aziende che comprano gli asset e li affidano a chi gioca e prendono le percentuali sulla vincita. 
%
% Nel metaverso di Meta non è previsto che ci siano NFT nè oggetti di proprietà. 
%
% New York NFT event. 
%
% La gente non capisce cosa sono gli NFT. L'immagine non è tanto importante. Il contenitore è più importante del contenuto. Il contratto che sta all'interno dell'NFT ti permette di entrare in eventi e locali e ti permette di interagire con altre persone che possiedono lo stesso NFT ossia un asset. Ma nel mondo è pieno di gente che vuole interagire ed è pieno di informazioni, servono quelli che sono i filtri: chi riesce a interagire ad un alto livello non è chi ha i contenuti giusti ma chi ha i filtri giusti. Io devo avere i filtri altrimenti mi arriva un sacco di filtro-smog non soddisfacente. Questa comincia ad essere una chiave di accesso che ti permette di interagire solamente con la gente che conosce l'ambiente. 
%
% Che ci fa un'azienda con un NFT? L'aspetto artistico di un NFT è secondario. Possono servire per creare una community e far parte di una community. È più facile per un'azienda monetizzare con gli NFT perché ad esempio possono esserci degli eventi o delle membership a cui gli utenti possono partecipare unicamente attraverso l'acquisto di NFT; un utente può essere incentivato ad essere membro (e quindi acquistare NFT) anche grazie a dinamiche di reward: chi fa oldare (detiene da più tempo l'NFT) ha diritto a dei premi esclusivi ad esempio. 
%
% Opensea: the largest NFT marketplace. Puoi interagire con Etherium ma le gas-fee sono altissime e quindi va da se che tutto ciò che vale di meno non verrà scambiato. Opensea è anche collegato con Polygon dove le fee sono praticamente gratis.
%
% Bitcoin asset finanziario paragonabile all'oro (un oro digitale) perché è la migliore tecnologia sul mercato ed è molto stabile. Etherium asset finanziario paragonabile al petrolio perché ti serve per fare cose ma è anche instabile visto che è una tecnologia nuova. 
%
% Ci sono due parametri per valutare un ecosistema di blockchain: quello della tecnologia e quello del network. La community che si è creata attorno ad etherium imparagonabile.